\documentclass[12pt]{article}
\usepackage[ruled,vlined,noresetcount]{algorithm2e}
\usepackage{amsmath}

\topmargin 0.0cm
\oddsidemargin 0.2cm
\textwidth 16cm
\textheight 21cm
\footskip 1.0cm

\begin{document}

\baselineskip24pt

\paragraph{Protein degradation}

\subparagraph{Model Implementation.}
The \texttt{ProteinDegradation} process accounts for the degradation of protein monomers. It uses the N-end rule \cite{Tobias:1991tz} to assign degradation rates for each protein, and selects proteins to be degraded as a Poisson process.\\

\begin{algorithm}[H]
\caption{Algorithm for Protein Degradation}
\SetKwInOut{Input}{Input}\SetKwInOut{Result}{Result}
\SetKwFunction{min}{min}
  
  \Input{$t_{1/2,i}$ Protein half-lives for each monomer where $i = 1$ \KwTo $n_{protein}$}
    \Input{$L_i$ length of each protein monomer where $i = 1$ \KwTo $n_{protein}$}
    \Input{$c_{aa,i,j}$ count of each amino acid present in the protein monomer where $i = 1$ \KwTo $n_{protein}$ and  $j = 1$ \KwTo $21$ for each amino acid}
    \Input{$c_{protein,i}$ the number of each protein present in the cell}
    \textbf{1.} Determine how many proteins to degrade based on the degradation rates and counts of each protein. \\
    \-\hspace{1cm} $n_{protein, i}$ = $\texttt{poisson}(\frac{ln(2)}{t_{1/2,i}} \cdot c_{protein, i} \cdot \Delta t$)\\
    \textbf{2.} Determine the number of hydrolysis reactions ($n_{rxns}$) that will need to occur.\\
    \-\hspace{1cm} $n_{rxns} = \sum\limits_i (L_i - 1) \cdot n_{proteins,i}$\\
    
    \textbf{3.} Determine the number of amino acids ($n_{aa,j}$) that will be released.\\
    \-\hspace{1cm} $n_{aa,j} = \sum\limits_i c_{aa,i,j} \cdot n_{proteins,i}$\\
    \textbf{4.} Degrade selected proteins, release amino acids from those proteins back into the cell, and consume $H_2O$ that was required for hydrolysis reactions.\\
    \textbf{Result:} Proteins are selected and degraded. During the process water is consumed, and amino acids are released. 
\end{algorithm}


\subparagraph{Difference from \textit{M. genitalium} model.}
The \emph{E. coli} model is not yet gene complete, hence this process does not take into account the activities of specific proteases and does not specifically target prematurely aborted polypeptides. In addition, protein unfolding and refolding by chaperones is not accounted for by this process. 


%\subparagraph*{Associated files}
\textbf{Associated files}

\begin{table}[h!]
 \centering
 \scriptsize
 \begin{tabular}{c c c} 
 \hline
 \texttt{wcEcoli} Path & File & Type \\
 \hline
\texttt{wcEcoli/models/ecoli/processes} & \texttt{protein\_degradation.py} & process \\
\texttt{wcEcoli/reconstruction/ecoli/dataclasses/process} & \texttt{translation.py} & data \\
 \hline
\end{tabular}
\caption[Table of files for protein degradation]{Table of files for protein degradation.}
\end{table}


%\subparagraph*{Associated data}
\textbf{Associated data}

 \begin{table}[h!]
 \centering
 \begin{tabular}{c c c c c} 
 \hline
 Parameter & Symbol & Units & Value & Reference \\
 \hline
 Protein half-lives & $t_{1/2}$ & min & [2, 600] & \cite{Tobias:1991tz} \\
 \hline
\end{tabular}
\caption[Table of parameters for protein degradation]{Table of parameters for protein degradation process.}
\end{table}

\newpage

\label{sec:references}
\bibliographystyle{plain}
\bibliography{supp_bib}

\end{document}